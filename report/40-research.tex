\section{Исследовательский раздел}


\subsection{Цель проводимых измерений}

Целью является исследование зависимости времени проверки валидности новой записи на прием от общего количества записей и места, где производятся вычисления: на уровне базы данных или приложения.

\subsection{Описание исследования}

При добавлении новой записи на прием необходимо проверить много факторов, например, что время для записи к врачу попадает в интервал его рабочих часов, питомец принадлежит клиенту, время начало раньше времени конца и выбранное время свободно. Эти проверки по-отдельности можно выполнять как на уровне базы данных, так и на уровне приложения. Также часть проверок можно вынести в интерфейс.

Для сравнения времени были выбраны проверки:
\begin{itemize}[label*=---]
	\item выбранное время для записи свободно;
	\item время начала и конца приема не противоречит рабочим часам доктора;
	\item прием длится ровно один час.
\end{itemize}

Также для исследования зависимости времени от объема таблицa c записями, количество записей будет увеличиваться от 50 до 1000. Можно заметить, что третий критерий не зависит от объема таблицы с записями на прием.

\subsection{Технические характеристики}

Исследование является четвертой стадией сборочной линии, описанной ранее. 

\subsection{Результаты исследования}
