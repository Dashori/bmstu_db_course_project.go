\section{Технологический раздел}

\subsection{Архитектура приложения}

Для приложения была выбрана клиент-серверная архитектура. Доступ к серверной части будет осуществляться с помощью API~\cite{api}. Для взаимодействия с базой данных будут использоваться коннекторы, предоставляющие интерфейс взаимодействия посредством языка программирования и, следовательно,  делать запросы из приложения.

\subsection{Средства реализации}

В рамках данной работы были выбраны следующие технологии.
\begin{enumerate}[label=---]
	\item Язык программирования -- Golang~\cite{go}.
	\item Система управления базами данных -- PostgreSQL~\cite{postgres}. 
	\item Для написания функций базы данных будет использовано расширение языка SQL -- PL/pgSQL~\cite{pgSQL}.
	\item Для работы с СУБД была выбрана библиотека database/sql ~\cite{go-sql}, предоставляющая универсальный интерфейс для реляционных баз данных (SQL), а также ее расширение -- библиотека sqlx~\cite{sqlx}.
	\item Фреймворк для реализации API -- gin~\cite{gin}.
	\item Для автоматизации развертывания и управления приложением была выбрана платформа Docker~\cite{docker}. 
	\item Технический интерфейс -- консоль. 
\end{enumerate}

\subsection{Детали реализации}


\subsubsection{Создание таблиц}




\subsubsection{Создание ролей на уровне базы данных}


В конструкторской части были выделены 4 роли на уровне базы данных: гость, клиент, доктор и администратор. Создание ролей и выделение им прав, в соответствии с ролевой моделью, представлены в листинге. 

\subsection{Хранимые функции}
3.3.1 Роли базы данных
В конструкторской части была разработана ролевая модель, в которой
выделены роли студента, преподавателя и администратора. Сценарий создания ролей и выделения им прав, соответствующих описанной ролевой модели,
представлен в приложении В.