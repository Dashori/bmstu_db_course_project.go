\section*{СПИСОК ИСПОЛЬЗОВАННЫХ ИСТОЧНИКОВ}
\addcontentsline{toc}{section}{СПИСОК ИСПОЛЬЗОВАННЫХ ИСТОЧНИКОВ}

\begingroup
\renewcommand{\section}[2]{}
\begin{thebibliography}{}
	
		\bibitem{businesstat}
		Маркетинговое агентство  BusinesStat, занимающееся исследованием конъюнктуры рынка [Электронный ресурс]. -- Режим доступа: 
		https://businesstat.ru/,
		свободный -- (12.03.2023)
		
		\bibitem{petstory}
		Ветеринарная клиника <<Petstory>> [Электронный ресурс]. -- Режим доступа:  https://petstory.ru/,
		свободный -- (12.03.2023)
		
		\bibitem{vetcity}
		Ветеринарная клиника <<Vetcity Clinic>> [Электронный ресурс]. -- Режим доступа:  https://vet.city/,
		свободный -- (12.03.2023)
		
		\bibitem{vetcare24}
		Ветеринарная клиника <<Vetcare24>> [Электронный ресурс]. -- Режим доступа: https://vetcare24.ru/,
		свободный -- (12.03.2023)
			
		\bibitem{Date}
		К. Дж. Дейт.
		<<Введение в системы баз данных>>, 8-е издание, издательский дом <<Вильяме>>,
		2005. -- 1238 c 
		
		\bibitem{Begg}
		Томас Коннолли, Каролин Бегг.
		<<Базы данных: Проектирование, реализация и сопровождение. Теория и практика>>, 3-е издание, издательский дом <<Вильяме>>,
		2017. -- 1440 c
		
		
		\bibitem{oracle}
		What is a Relational Database (RDBMS)? [Электронный ресурс]. -- Режим доступа: https://www.oracle.com/database/what-is-a-relational-database/, свободный -- (30.05.2023)
		

		\bibitem{sql}
		TECHNICAL SCIENCE / <<Colloquium-journal>> \#2(54),2020,
		Васильева К.Н., Хусаинова Г.Я,
		Реляционные базы данных,
		2020

		\bibitem{acid}
		What is ACID Compliance? [Электронный ресурс]. -- Режим доступа: 	https://www.mongodb.com/databases/acid-compliance,
		свободный -- (12.03.2023)
		
		\bibitem{amazon}
		What Is an In-Memory Database? | AWS Amazon [Электронный ресурс]. -- Режим доступа: https://aws.amazon.com/ru/nosql/in-memory/,
		свободный -- (12.03.2023)
		
		\bibitem{postgres}
		PostgreSQL: The World's Most Advanced Open Source Relational Database  [Электронный ресурс]. -- Режим доступа: https://www.postgresql.org/,
		свободный -- (12.03.2023)
		
		
		\bibitem{api}
		What is API?  [Электронный ресурс]. -- Режим доступа: https://www.ibm.com/topics/api свободный -- (12.03.2023)
		 
		
		\bibitem{go}
		Golang -- компилируемый многопоточный язык программированияe  [Электронный ресурс]. -- Режим доступа: https://pkg.go.dev/std/, свободный -- (12.03.2023)
		
		\bibitem{pgSQL}
		PL/pgSQL — SQL Procedural Language [Электронный ресурс]. -- Режим доступа: https://www.postgresql.org/docs/current/plpgsql.html, свободный -- (12.03.2023)
		
		
		\bibitem{go-sql}
		Библиотека database/sql [Электронный ресурс]. -- Режим доступа: https://pkg.go.dev/database/sql/,
		свободный -- (12.03.2023)
		
		\bibitem{sqlx}
		Библиотека sqlx [Электронный ресурс]. -- Режим доступа: https://pkg.go.dev/github.com/jmoiron/sqlx,
		свободный -- (12.03.2023)
		
		\bibitem{gin}
		Gin Web Framework. The fastest full-featured web framework for Go. [Электронный ресурс]. -- Режим доступа: https://gin-gonic.com/, свободный -- (12.03.2023)
		
		\bibitem{bcrypt}
		Package bcrypt implements Provos and Mazières's bcrypt adaptive hashing algorithm. [Электронный ресурс]. -- Режим доступа: https://pkg.go.dev/golang.org/x/crypto/bcrypt/, свободный -- (12.03.2023)
		
		\bibitem{bearer}
		Bearer authentication (also called token authentication) is an HTTP authentication scheme that involves security tokens called bearer tokens. [Электронный ресурс]. -- Режим доступа: https://swagger.io/docs/specification/authentication/bearer-authentication/, свободный -- (12.03.2023)
		
		\bibitem{docker}
		Docker is a platform designed to help developers build, share, and run modern application. [Электронный ресурс]. -- Режим доступа: https://docs.docker.com/, свободный -- (12.03.2023)
		
		
	 	\bibitem{ci}
		GitLab CI/CD is a tool for software development using the continuous methodologies: [Электронный ресурс]. -- Режим доступа: https://docs.gitlab.com/ee/ci,  свободный -- (30.05.2023)
		
		\bibitem{faker}
		Faker is a Python package that generates fake data for you.: [Электронный ресурс]. -- Режим доступа: https://faker.readthedocs.io/en/master/,  свободный -- (30.05.2023)
	
	
	

\end{thebibliography}
\endgroup

\pagebreak